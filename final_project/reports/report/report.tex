%%%%%%%%%%%%%%%%%%%%%%%%%%%%%%%%%%%%%%%%%
% Wenneker Article
% LaTeX Template
% Version 2.0 (28/2/17)
%
% This template was downloaded from:
% http://www.LaTeXTemplates.com
%
% Authors:
% Vel (vel@LaTeXTemplates.com)
% Frits Wenneker
%
% License:
% CC BY-NC-SA 3.0 (http://creativecommons.org/licenses/by-nc-sa/3.0/)
%
%%%%%%%%%%%%%%%%%%%%%%%%%%%%%%%%%%%%%%%%%

%----------------------------------------------------------------------------------------
%	PACKAGES AND OTHER DOCUMENT CONFIGURATIONS
%----------------------------------------------------------------------------------------

\documentclass[10pt, a4paper, twocolumn]{article} % 10pt font size (11 and 12 also possible), A4 paper (letterpaper for US letter) and two column layout (remove for one column)

%%%%%%%%%%%%%%%%%%%%%%%%%%%%%%%%%%%%%%%%%
% Wenneker Article
% Structure Specification File
% Version 1.0 (28/2/17)
%
% This file originates from:
% http://www.LaTeXTemplates.com
%
% Authors:
% Frits Wenneker
% Vel (vel@LaTeXTemplates.com)
%
% License:
% CC BY-NC-SA 3.0 (http://creativecommons.org/licenses/by-nc-sa/3.0/)
%
%%%%%%%%%%%%%%%%%%%%%%%%%%%%%%%%%%%%%%%%%

%----------------------------------------------------------------------------------------
%	PACKAGES AND OTHER DOCUMENT CONFIGURATIONS
%----------------------------------------------------------------------------------------

\usepackage[english]{babel} % English language hyphenation

\usepackage{microtype} % Better typography

\usepackage{amsmath,amsfonts,amsthm} % Math packages for equations

\usepackage[svgnames]{xcolor} % Enabling colors by their 'svgnames'

\usepackage[hang, small, labelfont=bf, up, textfont=it]{caption} % Custom captions under/above tables and figures

\usepackage{booktabs} % Horizontal rules in tables

\usepackage{lastpage} % Used to determine the number of pages in the document (for "Page X of Total")

\usepackage{graphicx} % Required for adding images

\usepackage{enumitem} % Required for customising lists
\setlist{noitemsep} % Remove spacing between bullet/numbered list elements

\usepackage{sectsty} % Enables custom section titles
\allsectionsfont{\usefont{OT1}{phv}{b}{n}} % Change the font of all section commands (Helvetica)

%----------------------------------------------------------------------------------------
%	MARGINS AND SPACING
%----------------------------------------------------------------------------------------

\usepackage{geometry} % Required for adjusting page dimensions

\geometry{
	top=1cm, % Top margin
	bottom=1.5cm, % Bottom margin
	left=2cm, % Left margin
	right=2cm, % Right margin
	includehead, % Include space for a header
	includefoot, % Include space for a footer
	%showframe, % Uncomment to show how the type block is set on the page
}

\setlength{\columnsep}{7mm} % Column separation width

%----------------------------------------------------------------------------------------
%	FONTS
%----------------------------------------------------------------------------------------

\usepackage[T1]{fontenc} % Output font encoding for international characters
\usepackage[utf8]{inputenc} % Required for inputting international characters

\usepackage{XCharter} % Use the XCharter font

%----------------------------------------------------------------------------------------
%	HEADERS AND FOOTERS
%----------------------------------------------------------------------------------------

\usepackage{fancyhdr} % Needed to define custom headers/footers
\pagestyle{fancy} % Enables the custom headers/footers

\renewcommand{\headrulewidth}{0.0pt} % No header rule
\renewcommand{\footrulewidth}{0.4pt} % Thin footer rule

\renewcommand{\sectionmark}[1]{\markboth{#1}{}} % Removes the section number from the header when \leftmark is used

%\nouppercase\leftmark % Add this to one of the lines below if you want a section title in the header/footer

% Headers
\lhead{} % Left header
\chead{\textit{\thetitle}} % Center header - currently printing the article title
\rhead{} % Right header

% Footers
\lfoot{} % Left footer
\cfoot{} % Center footer
\rfoot{\footnotesize Page \thepage\ of \pageref{LastPage}} % Right footer, "Page 1 of 2"

\fancypagestyle{firstpage}{ % Page style for the first page with the title
	\fancyhf{}
	\renewcommand{\footrulewidth}{0pt} % Suppress footer rule
}

%----------------------------------------------------------------------------------------
%	TITLE SECTION
%----------------------------------------------------------------------------------------

\newcommand{\authorstyle}[1]{{\large\usefont{OT1}{phv}{b}{n}\color{olive}#1}} % Authors style (Helvetica)

\newcommand{\institution}[1]{{\footnotesize\usefont{OT1}{phv}{m}{sl}\color{gray}#1}} % Institutions style (Helvetica)

\usepackage{titling} % Allows custom title configuration

\newcommand{\HorRule}{\color{olive}\rule{\linewidth}{1pt}} % Defines the gold horizontal rule around the title

\pretitle{
	\vspace{-30pt} % Move the entire title section up
	\HorRule\vspace{10pt} % Horizontal rule before the title
	\fontsize{32}{36}\usefont{OT1}{phv}{b}{n}\selectfont % Helvetica
	\color{teal} % Text colour for the title and author(s)
}

\posttitle{\par\vskip 15pt} % Whitespace under the title

\preauthor{} % Anything that will appear before \author is printed

\postauthor{ % Anything that will appear after \author is printed
	\vspace{10pt} % Space before the rule
	\par\HorRule % Horizontal rule after the title
	\vspace{20pt} % Space after the title section
}

%----------------------------------------------------------------------------------------
%	ABSTRACT
%----------------------------------------------------------------------------------------

\usepackage{lettrine} % Package to accentuate the first letter of the text (lettrine)
\usepackage{fix-cm}	% Fixes the height of the lettrine

\newcommand{\initial}[1]{ % Defines the command and style for the lettrine
	\lettrine[lines=3,findent=4pt,nindent=0pt]{% Lettrine takes up 3 lines, the text to the right of it is indented 4pt and further indenting of lines 2+ is stopped
		\color{olive}% Lettrine colour
		{#1}% The letter
	}{}%
}

\usepackage{xstring} % Required for string manipulation

\newcommand{\lettrineabstract}[1]{
	\StrLeft{#1}{1}[\firstletter] % Capture the first letter of the abstract for the lettrine
	\initial{\firstletter}\textbf{\StrGobbleLeft{#1}{1}} % Print the abstract with the first letter as a lettrine and the rest in bold
}

%----------------------------------------------------------------------------------------
%	BIBLIOGRAPHY
%----------------------------------------------------------------------------------------

\usepackage[backend=bibtex,style=authoryear,natbib=true]{biblatex} % Use the bibtex backend with the authoryear citation style (which resembles APA)
\addbibresource{reference.bib} % The filename of the bibliography


\usepackage[autostyle=true]{csquotes} % Required to generate language-dependent quotes in the bibliography

%----------------------------------------------------------------------------------------
%	ADD MY OWN
%----------------------------------------------------------------------------------------

\usepackage{listings}
\usepackage{xcolor}

\definecolor{codegreen}{rgb}{0,0.6,0}
\definecolor{codegray}{rgb}{0.5,0.5,0.5}
\definecolor{codepurple}{rgb}{0.58,0,0.82}
\definecolor{backcolour}{rgb}{0.95,0.95,0.95}
\definecolor{backcolour}{rgb}{0.95,0.95,0.95}
\definecolor{bluebox}{RGB}{166,211,255}

\lstdefinestyle{mystyle}{
    backgroundcolor=\color{backcolour},   
    commentstyle=\color{codegreen},
    keywordstyle=\color{codepurple},
    numberstyle=\tiny\color{codegray},
    stringstyle=\color{backcolor},
    basicstyle=\ttfamily\footnotesize,
    breakatwhitespace=false,         
    breaklines=true,                 
    captionpos=b,                    
    keepspaces=true,                 
    numbers=left,                    
    numbersep=5pt,                  
    showspaces=false,                
    showstringspaces=false,
    showtabs=false,                  
    tabsize=2
}

\lstset{style=mystyle}
\usepackage{hyperref}
\hypersetup{
    colorlinks=true,
    linkcolor=blue,
    citecolor=blue,
    filecolor=magenta,      
    urlcolor=blue,
    pdftitle={Overleaf Example},
    pdfpagemode=FullScreen,
}
\urlstyle{same}
 % Specifies the document structure and loads requires packages

%----------------------------------------------------------------------------------------
%	ARTICLE INFORMATION
%----------------------------------------------------------------------------------------

\title{Effective Methods for Capturing Cattle Rustlers} % The article title

\author{
	\authorstyle{Yung-Hsin Chen\textsuperscript{1} and Haoxin Cai\textsuperscript{1}} % Authors
	\newline\newline % Space before institutions
	\textsuperscript{1}\institution{Universit{\"a}t Z{\"u}rich, Z{\"u}rich, Switzerland} % Institution 1
}

% Example of a one line author/institution relationship
%\author{\newauthor{John Marston} \newinstitution{Universidad Nacional Autónoma de México, Mexico City, Mexico}}

\date{19 December 2022} % Add a date here if you would like one to appear underneath the title block, use \today for the current date, leave empty for no date

%----------------------------------------------------------------------------------------

\begin{document}

\maketitle % Print the title

\thispagestyle{firstpage} % Apply the page style for the first page (no headers and footers)

%----------------------------------------------------------------------------------------
%	ABSTRACT
%----------------------------------------------------------------------------------------

\lettrineabstract{The COVID-19 pandemic has severe impacts on almost every aspect around the world. 
It causes not only social and economic disruption, but also drastic death rates. Inevitably, people 
are curious about the causalities of COVID-19 and how to prevent the pandemic from getting worse. In 
this report, data are collected from 192 countries, and classification models are applied in order to get 
a list of feature importance. It is believed that the more-important-features play more crucial role
in the severity of the pandemic of a certain country.}

%----------------------------------------------------------------------------------------
%	ARTICLE CONTENTS
%----------------------------------------------------------------------------------------

%----------------------------------------------------------------------------------------
%	INTRODUCTION
%----------------------------------------------------------------------------------------
\section{Introduction}The goal of the report is to get the most related factors of COVID-19 
pandemic cases in countries. The time parameters are taken out of consideration for simplicity. 
(write more...)

%----------------------------------------------------------------------------------------
%	DOMAIN KNOWLEDGE
%----------------------------------------------------------------------------------------
\section{Domain Knowledge}\label{sec:domain_knowledge}

%----------------------------------------------------------------------------------------
%	METHOD
%----------------------------------------------------------------------------------------
\section{Method}
In this section, the methods of this particular task will be explained, including introduction of data, building 
features for the models, training and predicting classifiers and generating the feature importance table. 
Classifiers and other used tools are introduced and explained in \autoref{sec:domain_knowledge}.
\subsection{Data}
Data used for this task is from covid-19-data from Our World of Data. It is online in a github repository, making 
it easy to access. The data is loaded into the local database by the \emph{request} package of Python.\\[10pt]
The data consists of 67 columns and 248 countries with 236386 rows in total. Attribute \emph{location} and 
\emph{date} together define each unique row of data. The data is still being updated. 
A summary table of the data is shown 
in \autoref{tab:data_summary}.
\begin{table}
	\caption{Data Summary Table}
	\centering
	\begin{tabular}{lr}
		\toprule
		\textbf{Information} & \textbf{Description} \\
		\midrule
		number of columns & 67 \\
		number of rows & 236386 \\
		number of countries & 248 \\
		key & \{location:date\}\\
		starting date & 01.Jan.2020\\
		\bottomrule
	\label{tab:data_summary}
	\end{tabular}
\end{table}
\subsection{Building Features}
The process of building features includes data cleaning and feature selection, and label preparation. Data cleaning 
and feature selecting is crucial for the accuracy of models. Bad data cleaning can lead to biased results or 
bad model performances. Relevant attributes will be selected as features to be put into the classifiers, i.e., 
the models. Finally, since this is a supervised learning task, the label should be prepared for model training.\\[10pt]
In the data cleaning phase, the goal is to get a table of one row per country, i.e., each row represents the 
information of a country. The label is defined as the attribute, \emph{total\_cases\_per\_million}. To achieve this, 
relavent attributes are first selected for model training. For this task, eleven attributes that is speculated to 
affect the number of COVID cases are selected from the raw data. The selected attributes are listed in 
\autoref{tab:attributes}. Countries with less than 200 rows of data is then removed. Since each row corresponds 
to one date, it is not ideal to use the data of a country if less than 200 days of data are recorded. Among the 
selected attributes, \emph{aged\_65\_older} and \emph{aged\_70\_older} are divided by \emph{population} into 
\emph{aged\_65\_older\_percentage} and \emph{aged\_70\_older\_percentage} respectively so that the models are 
trained on the percentage of elder people instead of the total number. Except for the attribute 
\emph{people\_fully\_vaccinated\_per\_hundred}, all the other attributes have a single value throughout the dates 
for each country. However, the attribute \emph{people\_fully\_vaccinated\_per\_hundred} and the label attribute 
\emph{total\_cases\_per\_million} is accumulated day by day. In this case, the data of the latest date is used as 
the feature value for each country. By doing this, the model will be able to generate the feature importance 
table according to how all features affect the number of total cases per million in each country. After data 
cleaning and feature selection, 194 countries/rows and 11 features are left for model training.\\[10pt]
\begin{table}
	\caption{Data Summary Table}
	\centering
	\begin{tabular}{ll}
		\toprule
		\textbf{Selected Attributes} &  \\
		\midrule
		aged\_65\_older & cardiovasc\_death\_rate \\
		aged\_70\_older & diabetes\_prevalence \\
		gdp\_per\_capita & hospital\_beds\_per\_thousand \\
		population\_density & human\_development\_index \\
		life\_expectancy & median\_age \\
		\multicolumn{2}{l}{people\_fully\_vaccinated\_per\_hundred} \\
		\bottomrule
	\label{tab:attributes}
	\end{tabular}
\end{table}
After data cleaning and feature selection, label preparation is performed. The attribute \emph{total\_cases\_per\_million} 
is categorised into four levels of severity. The interval of the categorisation is shown in \autoref{tab:label_interval}. 
Apparently, there is no country with over 700'000 cases per million. 
\begin{table}
	\caption{Label Interval for Label Preparation}
	\centering
	\begin{tabular}{ll}
		\toprule
		\textbf{Level} & \textbf{Interval} \\
		\midrule
		0 & 0 - 50'000 \\
		1 & 50'000 - 200'000 \\
		2 & 200'000 - 400'000 \\
		3 & 400'000 - 700'000\\
		\bottomrule
	\label{tab:label_interval}
	\end{tabular}
\end{table}
[0, 50000, 200000, 400000, 700000]
\subsection{Classifiers}
Classifiers are used for classification tasks. Due to the small dimension of the dataset, several changes are made to 
adapt the data size. The data is not splitted into training, validation and testing datasets, but only training and 
testing only. In this task, the training-testing split will be 80\% and 20\% respectively. Besides, it is not recommended 
to use models with too many parameters since it might have to higher chance of overfitting. Thus, simple models are 
picked out for this task including logistic regression, linear perceptron and XGBoost. These models can be easily 
applied to the dataset with \emph{sklearn} from Python. For hyperparameter selection, grid search cross validation 
is used. A summary of models used for the task and the best performing hyperparameters chosen by applying grid search 
cross validation are listed out in \autoref{tab:model_summary}.
\begin{table}
	\caption{Model Summary}
	\centering
	\begin{tabular}{lr}
		\toprule
		\textbf{Model} & \textbf{Details} \\
		\toprule
		\textbf{logistic regression} & \\ 
		penalty & l2 \\
		solver & newton-cg \\
		\midrule
		\textbf{Linear Perceptron} & \\
		tolerance & 0.001 \\
		penalty & l2 \\
		\midrule
		\textbf{XGBoost} & \\
		learning rate & 0.1\\
		loss & deviance \\
		max depth & 3 \\
		n\_estimators & 100 \\
		random state & 21 \\
	\bottomrule
	\label{tab:model_summary}
	\end{tabular}
\end{table}
\subsection{Feature Importance Table}
The feature importance is generated automatically via the trained model. It shows weight of each feature. The weight can 
be thought of as how significant each feature effects the classification accuracy. 
\section{Results}
\section{Dicussion}
\section{Conclusion}
%----------------------------------------------------------------------------------------
%	BIBLIOGRAPHY
%----------------------------------------------------------------------------------------

\printbibliography[title={Bibliography}] % Print the bibliography, section title in curly brackets

%----------------------------------------------------------------------------------------

\end{document}
